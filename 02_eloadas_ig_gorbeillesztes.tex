\documentclass{beamer}
\usepackage[magyar]{babel}
\usepackage[T1]{fontenc}
\usepackage{hyphenat}
\usepackage{amsmath}

\title{Görbeillesztés}
\subtitle{Lineáris regresszió, legkisebb négyzetek módszere}
\author{Illés Gergő és Sarkadi Balázs}
\institute{PTE TTK Fizikai Intézet}

\begin{document}
\begin{frame}
\titlepage
\end{frame}
\begin{frame}
\frametitle{Tartalom}
\tableofcontents
\end{frame}
\section{Lineáris regresszió}
\subsection{Általános ismertetés}
\begin{frame}
\frametitle{\subsecname}
\begin{itemize}
\item A lineáris regresszió a leggyakrabban használt görbeillesztési módszer.
\item Lineáris regresszió használatakor lineáris kapcsolatot feltételezünk a független\hyp{} és függő változó között.
\item Nemlineáris kapcsolatokra is alkalmazható a függő\hyp{} és független változók közti kapcsolat linearizálásával.
\end{itemize}
\end{frame}
\begin{frame}
\frametitle{\subsecname}
Tételezzük fel, hogy rendelkezünk egy adathalmazzal amely $n$ darab statisztikai egységet tartalmaz. Ezt mátrix formájában a következőképp írhatjuk le.
\begin{equation*}
\left[
\begin{matrix}
\left\lbrace y_1,[x_{11},\dots,x_{1p}]\right\rbrace\\
\vdots\\
\left\lbrace y_n,[x_{n1},\dots,x_{np}]\right\rbrace\\
\end{matrix}
\right]
\end{equation*}
Ebben az írásmódban $y$ a függő változó, $\vec{x}$ $p$ hosszúságú vektor pedig az úgynevezett regresszor ami a független változókat tartalmazza.
\end{frame}
\begin{frame}
\frametitle{\subsecname}
A becslésünk jóságára vezessünk be egy hibaváltozót, ez legyen $\epsilon$. Lineáris függést, valamint $\epsilon$ hibát feltételezve az egyes $y_i$-k a következőképp írhatók fel:
\begin{align*}
y_i&=\beta_0+\sum_{j=1}^{p}\beta_jx_{ij}+\epsilon_i
\end{align*}
\end{frame}
\end{document}